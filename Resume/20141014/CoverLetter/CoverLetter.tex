%% start of file `template.tex'.
%% Copyright 2006-2013 Xavier Danaux (xdanaux@gmail.com).
%
% This work may be distributed and/or modified under the
% conditions of the LaTeX Project Public License version 1.3c,
% available at http://www.latex-project.org/lppl/.


\documentclass[10pt,a4paper,sans]{moderncv}        % possible options include font size ('10pt', '11pt' and '12pt'), paper size ('a4paper', 'letterpaper', 'a5paper', 'legalpaper', 'executivepaper' and 'landscape') and font family ('sans' and 'roman')

% moderncv themes
\moderncvstyle{banking}                             % style options are 'casual' (default), 'classic', 'oldstyle' and 'banking'
\moderncvcolor{grey}                               % color options 'blue' (default), 'orange', 'green', 'red', 'purple', 'grey' and 'black'


% adjust the page margins
\usepackage[scale=0.80, margin=0.5in]{geometry}
%\setlength{\hintscolumnwidth}{3cm}                % if you want to change the width of the column with the dates
%\setlength{\makecvtitlenamewidth}{10cm}           % for the 'classic' style, if you want to force the width allocated to your name and avoid line breaks. be careful though, the length is normally calculated to avoid any overlap with your personal info; use this at your own typographical risks...

% personal data
\name{Aiswarya}{Kolisetty}
%\address{429 N Center St}{Royal Oak, MI}
\phone[mobile]{(617)~823~7803}                   % optional, remove / comment the line if not wanted; the optional "type" of the phone can be "mobile" (default), "fixed" or "fax"
\email{aiswaryakolisetty@gmail.com}                               % optional, remove / comment the line if not wanted
\homepage{www.aiswaryakolisetty.com}                         % optional, remove / comment the line if not wanted
%\social[linkedin]{aiswaryakolisetty}                        % optional, remove / comment the line if not wanted
%\quote{Some quote}                                 % optional, remove / comment the line if not wanted


\begin{document}

% recipient data
\recipient{GE Recruiting}{}
\date{October 10, 2014}
\opening{Dear Sir or Madam,}
\closing{Thank you for your time,}
%\enclosure[Attached]{curriculum vit\ae{}}          % use an optional argument to use a string other than "Enclosure", or redefine \enclname
\makelettertitle

I would like to be considered for the user experience engineer position at GE.  I am interested in working with a cross-functional team of managers, end users, and business leaders to develop delightful experiences for a new product.  

At my current job at Ford Research (close by, in Dearborn), I have created my own version of 'agile' UX development in which I produce wireframes or UI prototypes for my team every 2 weeks.  I am eager to learn how GE does agile development and figure out how my techniques can fit in.  As I create UI prototypes, I am using basic AngularJS directives, and I would like to use this opportunity at GE to develop more in-depth skills in Javascript frameworks. It'll help me build faster and cleaner prototypes too.  
\smallskip

 From a skills standpoint, I believe I would be a good fit for the position because:


1. I am a UX designer with experience in user research, wireframing, front-end UI prototyping, usability testing, writing requirements;

2. I believe in a user-driven design approach, where the needs and wants of the user base should provide a foundation for building a product;

3.  My favorite part of the process is in the creative side - sketching, creating paper prototypes, interviewing users, and translating my results into digital Illustrator concepts;

4. I enjoy communicating my design process to project stakeholders and I strive to support my design decisions with qualitative (interviews) or quantitative (survey) data;

5. I experiment with new devices, design techniques, and APIs in the market to see how they can fit into my personal and professional projects;

6. I am developing my front-end development skills to prototype on my own and  know the limitations to my design;

7. I am always pushing myself to learn more and provide new value to my teams and projects.

\bigskip

You can see examples of my work at www.aiswaryakolisetty.com/work.html.  I maintain a weekly blog to consolidate life lessons and to improve my writing at www.aiswaryakolisetty.com/blog.html.    

I would love the opportunity to speak with you and discuss how my skillset can help enhance the user experience of GE's products.


\bigskip

\makeletterclosing

%\clearpage\end{CJK*}                              % if you are typesetting your resume in Chinese using CJK; the \clearpage is required for fancyhdr to work correctly with CJK, though it kills the page numbering by making \lastpage undefined
\end{document}


%% end of file `template.tex'.
