%% start of file `template.tex'.
%% Copyright 2006-2013 Xavier Danaux (xdanaux@gmail.com).
%
% This work may be distributed and/or modified under the
% conditions of the LaTeX Project Public License version 1.3c,
% available at http://www.latex-project.org/lppl/.


\documentclass[10pt,a4paper,sans]{moderncv}        % possible options include font size ('10pt', '11pt' and '12pt'), paper size ('a4paper', 'letterpaper', 'a5paper', 'legalpaper', 'executivepaper' and 'landscape') and font family ('sans' and 'roman')

% moderncv themes
\moderncvstyle{banking}                             % style options are 'casual' (default), 'classic', 'oldstyle' and 'banking'
\moderncvcolor{grey}                               % color options 'blue' (default), 'orange', 'green', 'red', 'purple', 'grey' and 'black'


% adjust the page margins
\usepackage[scale=0.80, margin=0.5in]{geometry}
%\setlength{\hintscolumnwidth}{3cm}                % if you want to change the width of the column with the dates
%\setlength{\makecvtitlenamewidth}{10cm}           % for the 'classic' style, if you want to force the width allocated to your name and avoid line breaks. be careful though, the length is normally calculated to avoid any overlap with your personal info; use this at your own typographical risks...

% personal data
\name{Aiswarya}{Kolisetty}
%\address{429 N Center St}{Royal Oak, MI}
\phone[mobile]{(617)~823~7803}                   % optional, remove / comment the line if not wanted; the optional "type" of the phone can be "mobile" (default), "fixed" or "fax"
\email{aiswaryakolisetty@gmail.com}                               % optional, remove / comment the line if not wanted
\homepage{www.aiswaryakolisetty.com}                         % optional, remove / comment the line if not wanted
%\social[linkedin]{aiswaryakolisetty}                        % optional, remove / comment the line if not wanted
%\quote{Some quote}                                 % optional, remove / comment the line if not wanted


\begin{document}

% recipient data
\recipient{Amazon}{UX Designer}
\date{May 25, 2015}
\opening{}
\closing{Thank you for your time.}
%\enclosure[Attached]{curriculum vit\ae{}}          % use an optional argument to use a string other than "Enclosure", or redefine \enclname
\makelettertitle

"I have to use my brain for it and that doesn't make me feel like using it. Paper is better."  This is my mother's reaction to technology.  Two weeks ago she started using a Mac and an iPhone for the first time.  And boy, has that changed my life.

I'm sitting beside my mother on the couch, each to our own laptops.  She looks over and gives me  "the look" and asks me to teach her a new skill.  I groan, but I'm silently impressed at how well she's overcoming her fear of the internet.  The early days were tough, and she would come up with some unbelievable questions.  Why is Ctrl+C for copy and Ctrl+V for paste and who thought of that?  How come the web browser lets me search for an answer and type a website link in the same box?  As if I knew!  Certainly, I can't blame her for those questions.  But they weren't the task-based questions she ought to be asking. She doesn't know what to ask. 

The exposure I'm getting to my mother's struggles is lesson in user experience.  She is a late adopter and that's rare nowadays.  She embodies a persona that lives all around the world.  A persona that touches products that I touch and experiences situations that I do.  But she does it with a distance and aversion from technology and it's my job to understand what about it works for her.  And what doesn't work for her.  It's my job to help her draw the line on how comfortable she is with using technology.    

How can I help her and help people like her?  It's not a scalable solution to demand she memorize the steps (clearly it hasn't worked in the education industry), but it's by designing aides that guide her in the right path.  For example, Gmail provides a starter guide when you first open Gmail with tooltips labeling each section on the screen. Let us keep it enabled in a "novice mode" till the email process becomes natural to her.  What can Gmail do to avoid losing my mom's intention to add recipients to her email?

A lot of my user experience lessons have come from my design classes in college and my job at Ford.  But being a designer of experiences, I get a lot of unexpected lessons outside the workplace too.  My own family teaches me how technology scares or inhibits them. My mother teaches me why she would pick up the pen and paper without blinking an eye.  She practices technology avoidance even when she knows that it comes with benefits.  

User experience is about these immersive situations.  It's about noticing my mother's glance as she gets stuck on another Safari page.  It's about noticing what she tries and how she decides to give up on sending that email.  It's about asking her what she tried, why she ended up here, what made her take this path.  

When a company's product lives and breathes on the internet, how can it positively touch the lives of people like my 45 year old mother?  She can't dream of buying something online, let alone search for a new vacuum machine and read the reviews.  How do we make a positive impact on her? What steps can we take at showing her that she can order the vacuum cleaner online and have it delivered to her door?  How can we help her transition from technology avoidance to awareness, from fear to curiosity?

My hope is to synthesize lessons like this one into my design thinking, as I work for a company like Amazon.  It's actually been a breath of fresh air to teach my mom how to get started on technology. I plan to take her lessons with me in design thinking, ensuring that I don't leave others like her behind.  

I look forward to your response.  Speaking of which, I think Amazon will be the next app I help my mother install on her phone this week...

\bigskip

\makeletterclosing

%\clearpage\end{CJK*}                              % if you are typesetting your resume in Chinese using CJK; the \clearpage is required for fancyhdr to work correctly with CJK, though it kills the page numbering by making \lastpage undefined


\end{document}


%% end of file `template.tex'.
