%% start of file `template.tex'.
%% Copyright 2006-2013 Xavier Danaux (xdanaux@gmail.com).
%
% This work may be distributed and/or modified under the
% conditions of the LaTeX Project Public License version 1.3c,
% available at http://www.latex-project.org/lppl/.


\documentclass[10pt,a4paper,sans]{moderncv}        % possible options include font size ('10pt', '11pt' and '12pt'), paper size ('a4paper', 'letterpaper', 'a5paper', 'legalpaper', 'executivepaper' and 'landscape') and font family ('sans' and 'roman')

% moderncv themes
\moderncvstyle{banking}                             % style options are 'casual' (default), 'classic', 'oldstyle' and 'banking'
\moderncvcolor{grey}                               % color options 'blue' (default), 'orange', 'green', 'red', 'purple', 'grey' and 'black'


% adjust the page margins
\usepackage[scale=0.80, margin=0.5in]{geometry}
%\setlength{\hintscolumnwidth}{3cm}                % if you want to change the width of the column with the dates
%\setlength{\makecvtitlenamewidth}{10cm}           % for the 'classic' style, if you want to force the width allocated to your name and avoid line breaks. be careful though, the length is normally calculated to avoid any overlap with your personal info; use this at your own typographical risks...

% personal data
\name{Aiswarya}{Kolisetty}
%\address{429 N Center St}{Royal Oak, MI}
\phone[mobile]{(617)~823~7803}                   % optional, remove / comment the line if not wanted; the optional "type" of the phone can be "mobile" (default), "fixed" or "fax"
\email{aiswaryakolisetty@gmail.com}                               % optional, remove / comment the line if not wanted
\homepage{www.aiswaryakolisetty.com}                         % optional, remove / comment the line if not wanted
%\social[linkedin]{aiswaryakolisetty}                        % optional, remove / comment the line if not wanted
%\quote{Some quote}                                 % optional, remove / comment the line if not wanted


\begin{document}

% recipient data
\recipient{Think Design}{User Experience Designer, Bangalore}
\date{May 25, 2015}
\opening{Hello,}
\closing{Thank you for your time.}
%\enclosure[Attached]{curriculum vit\ae{}}          % use an optional argument to use a string other than "Enclosure", or redefine \enclname
\makelettertitle

My name is Aiswarya and I currently work at Ford Motor Company.  Here, I've been a user experience designer in R\&D for two years and I've worked with several design consultancies for an outside perspective.  Consultancies have helped us answer the question "Do people really care about owning a car anymore?"  And if not, why not? What can we do differently?
 
In this codesign experience, I've come to understand that a consultancy can be a bane or boon to us.  A boon when they are able to get out into the real world, live and spend time with every day people in places like India, Brazil, or France.  They gather rich ethnographic material to help us understand the people outside the Ford bubble.  The people who don't see our company as a business, but an everyday product they can do with or without.  

A consultancy may end up a bane if they aren't ready to follow the pace, process, or product strategy of our company.  They come with currently-unattainable ideas and speak about far-in-the-future goals.  At that point, we could just do with a book report.
 
My experience on the other side of the table can be a valuable asset if I get to work at Think Design.  Understanding the business, goals from the study, mindset on a design exploration - these elements are building blocks of the client.  They need to be understood before a project begins.  Only if we can speak the same language as our client, we can make respectful progress.  

At Ford and at previous internships, I've done ethnographic studies, persona development, storyboarding, and storytelling of my ideas.  In Ford Research, it was about selling our ideas on the future car via rapid prototypes on web and mobile.  My favorite tools are post-its, Balsamiq, and Bootstrap.  My desk doesn't always fit in with the cubicle environment; it's full of scrap paper, sharpies, and design ideas.

My resume is attached and my ongoing and previous work can be found at www.aiswaryakolisetty.com.

I'm very interested in working at Think Design as a UX Designer and engaging in a broad spectrum of client needs and wants.  I look forward to your reply.


\bigskip

\makeletterclosing

%\clearpage\end{CJK*}                              % if you are typesetting your resume in Chinese using CJK; the \clearpage is required for fancyhdr to work correctly with CJK, though it kills the page numbering by making \lastpage undefined


%%%%%%%%%%%%%%%%%%% RESUME %%%%%%%%%%%%%%%%%%%%
%%%%%%%%%%%%%%%%%%%%%%%%%%%%%%%%%%%%%%%%%%%
\clearpage

% adjust the page margins
%\setlength{\hintscolumnwidth}{3cm}                % if you want to change the width of the column with the dates
%\setlength{\makecvtitlenamewidth}{10cm}           % for the 'classic' style, if you want to force the width allocated to your name and avoid line breaks. be careful though, the length is normally calculated to avoid any overlap with your personal info; use this at your own typographical risks...

% personal data
\name{Aiswarya}{Kolisetty}
%\address{429 N Center St}{Royal Oak, MI}
\phone[mobile]{(617)~823~7803}                   % optional, remove / comment the line if not wanted; the optional "type" of the phone can be "mobile" (default), "fixed" or "fax"
\email{aiswaryakolisetty@gmail.com}                               % optional, remove / comment the line if not wanted
\homepage{www.aiswaryakolisetty.com}                         % optional, remove / comment the line if not wanted
%\social[linkedin]{aiswaryakolisetty}                        % optional, remove / comment the line if not wanted
%\quote{Some quote}                                 % optional, remove / comment the line if not wanted


\makecvtitle


\section{Experience}
\smallskip
\cventry{July 2013 - Present}{User Experience Engineer}{Ford Motor Company R\&D}{Dearborn, MI}{}
{On contract from Kelly Services starting July 2014}
{{}%
\smallskip
\begin{itemize} 
	
	\item Leading the UI design of 6 driver-centric connectivity features 
	\item Developing concepts with personas, storyboards, wireframes, and Balsamiq mockups 
	\item Conducting iterative user research and usability tests with over 50 individuals on multiple projects
	\item  Implementing front-end prototypes using HTML, CSS, JavaScript, JSON/XML, external APIs, cloud data
	\item Working with external design firms on synthesizing global ethnographic research
	\item Trying to make an individual impact at a large company
	\bigskip
	\item Received 2 employee recognition awards in 2014
	\item Main inventor on 1 US Patent filing, submitted over 15 inventions disclosures internally
	\item Designed the UX and UI specifications for two features currently at production
	\item Led and launched our first departmental Ideation Session, resulting in hundreds of project and invention ideas
	
\end{itemize}}

\bigskip
\subsection{Internships}
\bigskip
\cventry{Feb - May 2013}{User Experience Consultant}{Scargo Inc.}{Waltham, MA}{}
{{}%
\begin{itemize} 
	\item  Conducted usability tests on the Scargo product and provided UI changes to the on-boarding experience
\end{itemize}}

\cventry{May - August 2012}{Model S Systems Test Intern}{Tesla Motors}{Palo Alto, CA}{}
{{}%
\begin{itemize} 
	\item  Automated 2+ hours of vehicle testing for weekly firmware release, written and tested in C language
	\item  Performed internal usability tests on the Model S UI and wireframed new designs
\end {itemize}}

\cventry{June-August 2011}{Vehicle Design Intern}{Imec}{Leuven, Belgium}{}
{{}%
\begin{itemize} 
	\item  Created a concept vehicle, {\emph{Aura}}, for young commuters in European cities
	\item   Designed exterior, interior, and drivetrain specifications for Aura
\end{itemize}}




%\cventry{year--year}{Job title}{Employer}{City}{}{Description line 1\newline{}Description line 2}
\bigskip

\section{Education}
\smallskip
\cventry{2009 - 2013}{B.Sc Electrical and Computer Engineering}{Franklin W. Olin College of Engineering}{Needham, MA}{{(concentration in design)}}
{GPA: 3.50}  % arguments 3 to 6 can be left empty

\bigskip

\subsection{Projects}
\cventry{Sept 2012 - May 2013}{3D modeling in secondary education}{}{}{}
{Engineering capstone project with Autodesk Inc.}
{{}%
\begin{itemize} 
 	\item Designed \textit{TechBits}, small design activities that empower students to use CAD and design solutions
	 \item Deliverables: STEAM user research, 4 sample TechBits, framework to make a TechBit, usability testing results
\end{itemize}}

\cventry{Sept 2012 - May 2013}{Aging Research Group}{}{}{}
{}
{{}%
\begin{itemize} 
 	\item Conducted usability experiments on the use of smart devices in the Baby Boomer generation
	 \item Redesigned the UI for an iOS app that helps aging adults with mild memory loss
\end{itemize}}

%\cventry{Sept - Dec 2011}{BOWtie: A course planning tool for college}{}{}{}
%{Human Factors Interface Design class}
%{{}%
%\begin{itemize} 
%	\item Designed \textit{BOWtie} using user research, paper prototypes, heuristic evaluations, usability testing
%	\item Built the interactive course-planning tool in HTML5, CSS, JavaScript, jQuery
%\end{itemize}}


\bigskip
\section{Skills and Tools}
\smallskip
\cvdoubleitem{Design}{Balsamiq, Visio, Illustrator, Photoshop, Axure} {Other}{C\#, C++,  Python, Java, Objective-C (basic)}
\cvdoubleitem{Front-end}{HTML, CSS, JS, Bootstrap,  AngularJS }{}{}

%\section{Interests}
%\cvitem{}{Volleyball, photography, writing}


%\section{References}
%\begin{cvcolumns}
%\cvcolumn{Category 1}{\begin{itemize}\item Person 1\item Person 2\item Person 3\end{itemize}}
%  \cvcolumn{Category 2}{Amongst others:\begin{itemize}\item Person 1, and\item Person 2\end{itemize}(more upon request)}
 % \cvcolumn[0.5]{All the rest \& some more}{\textit{That} person, and \textbf{those} also (all available upon request).}
%\end{cvcolumns}


\end{document}


%% end of file `template.tex'.
